\chapter{Grundlagen}
\label{sec:grundl}


\subsubsection{Adafriut Ultimate GPS PI HAT\cite{doc:gpsHat}}
Dieses Modul wird direkt auf das PI aufgesteckt und leitet alle Pins bis auf die UART Pins weiter. Folgende Pins werden f�r das Bord ben�tigt und k�nnen nicht mehr verwendet werden:

\begin{table}[H]
	 \centering
	\begin{tabular}{|p{3cm}|p{11cm}| p{0.7cm} |}
		\hline 
		Pin &  Verwendung & Fest\footnotetext{Beschaltung kann nicht ge�ndert werden} \\ 
		\hline 
		UART TXD ; UART RXD & Die einzige serielle Schnittstelle muss verwendet werden um mit dem GPS-Modul zu kommunizieren \\
		\hline 
		GPIO \#4 &  Kann bei Bedarf verwendet werden, falls die Zeitsynchronisation mit dem Raspberry nicht ben�tigt wird \\
		\hline
		EEDATA ;  EECLK & Werden f�r die Verbindung mit dem EEPROM ben�tigt, derzeit noch nicht vom Raspberry verwendet  \\
		\hline
	\end{tabular}
	\caption{Ultimate GPS Pins}
\end{table}

Das Modul bietet Platz f�r eine Batteriezelle, f�r eine Real-Time-Clock und Prototypen Platz, um weitere Bauteile mittels L�ten hinzuzuf�gen. 

\subsubsection{ADS1015 12-Bit ADC\cite{doc:gpsADC}}

Die Standardadresse des Bausteines ist die 7bit-Adresse 0x47. Diese Adresse kann durch die Verbindung folgender Pins mit dem \emph{ADDR-Pin} bei Bedarf angepasst werden: 
	
\begin{table}[H]
	\centering
	\begin{tabular}{|p{4cm}|p{4cm}| p{4cm} |}
		\hline 
		Adresse &  Pin 1 & ADDR\\ 
		\hline 
		0x48 &  GND &ADDR\\ 
		\hline 
		0x49 &  VDD & ADDR\\ 
		\hline 
		0x4A &  SDA & ADDR\\ 
		\hline 
		0x4B &  SCL & ADDR\\ 
		\hline 
	\end{tabular}
	\caption{ADC 12 Bit I2C Adress Manipulation}
\end{table}

Dieser ADC liefert folgende zwei verschiedene Betriebsmodi, welche �ber die Anschl�sse \emph{A0} bis \emph{A3} verwendet werden:


\begin{itemize}
	\item \emph{Single Ended}: berechnet den ADC der Spannung zwischen jedem der \emph{Ax-Pins} und Ground. Hiermit ist nur das Messen von positiven Spannungen m�glich. Effektiv verliert man dadurch ein Bit der Aufl�sung. Der Modus bietet doppelt so viele Inputs.
	\item \emph{Differential}: berechnet den ADC der Spannung zwischen \emph{A0} \& \emph{A1} und \emph{A2} \& \emph{A3}. Der Modus liefert weniger rauschanf�llige Signale.
\end{itemize}


Eine Spannung �ber 5V darf nicht angelegt werden, da diese das Modul zerst�ren w�rde.

\subsubsection{Pololu AltIMU-10 v4\cite{doc:imu}}

Dieses Modul besteht aus drei IC�s. Sie beinhalten die folgenden Funktionen:

\begin{table}[H]
	\centering
	\begin{tabular}{|p{4cm}|p{3cm}|p{4cm}|p{3.5cm}|}
	\hline
	 Funktion & IC & Default I2C Adresse  & Manipulate I2C  \\
	 \hline
	 Beschleunigungsmesser \newline  Magnetometer & \href{https://www.pololu.com/file/download/LSM303D.pdf?file_id=0J703}{LSM303D}  & 0x1D & 0x1E\\
	  \hline
	   Gyrometer &
	  \href{https://www.pololu.com/file/download/L3GD20H.pdf?file_id=0J731}{L3GD20H}& 0x6B& 0x6A \\
	  \hline
	 Barometer& \href{https://www.pololu.com/file/download/LPS25H.pdf?file_id=0J761}{LPS25H }& 0x5D & 0x5C \\
	 \hline 
	\end{tabular}
	\caption{IMU ICs}
\end{table}

Auch dieser Baustein verwendet zur Konfiguration und Kommunikation den I2C-Bus (7bit Adresse). Jeder dieser ICs besitzt eine eigene Adresse, die sich ebenfalls manipulieren l�sst, indem man den Pin \emph{SA0} auf \emph{GND} zieht.

\subsubsection{LIDAR-Lite v2\cite{doc:lidar}}

Dieser Lasersensor besitzt eine Reichweite bis zu $ 40 m$ mit einer Genauigkeit von $  \pm 0,025$ , eine Verz�gerung von 0,02 Sekunden und kann bis zu $500$ Messwerte pro Sekunde liefern. Der I2C-Bus kann mit $100kbits/s$ oder $400kbits/s$ betrieben werden. Eine eigene Adressierung ist m�glich. Als Standard-I2C Adresse ist die 0x62 festgelegt. Falls mehrere LIDAR Light v2 Sensoren an einem I2C-Bus h�ngen ist dies die Broadcast-Adresse.

