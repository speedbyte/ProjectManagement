
%%%%%%%%%% Kopfbereich %%%%%%%%%%%%%
%\documentclass[a4paper,titlepage,oneside,fontsize=2pt]{scrbook} % Dokumentklasse
\documentclass[11pt,a4paper,titlepage,oneside]{report} % Dokumentklasse
\pdfminorversion=7  %akzeptiert pdf in version 1.7
%\RequirePackage{pdf14}
\usepackage{etex}

\usepackage{nomencl,longtable,ifthen} % Symbolverzeichnis
%\usepackage{ngerman} % Deutsch, neue Rechtschreibung
%\usepackage[latin1]{inputenc} % Sonderzeichen ��� etc.
\usepackage[T1]{fontenc} % T1 Format
\usepackage{geometry} % Seitenr�nder
\usepackage[printonlyused]{acronym} % Abk�rzungsverzeichnis
\usepackage{graphicx} % Einbinden von Grafiken
\usepackage{fancyhdr} % Gestaltung von Kopf- und Fu�zeile
%\usepackage{bibgerm} % Literaturverzeichnis
%\usepackage[squaren]{SIunits} % SI Einheiten
\usepackage{amsmath, amsthm, amssymb}
\usepackage{mathtools}
\usepackage{floatflt}
\usepackage{float}
\usepackage{graphics}
\usepackage{picins}
\usepackage{wrapfig}
\usepackage{threeparttable}
\usepackage{textcomp}
\usepackage{tabularx}
\usepackage{hhline}
\usepackage{siunitx}
\usepackage{framed, color}

\usepackage{courier}
\usepackage{listings}
\usepackage{color}
\usepackage{multicol}
\usepackage{multirow}
\usepackage[hyphens]{url}
\usepackage{xcolor}
\usepackage[right]{eurosym}
\usepackage{colortbl}
\usepackage{subfigure}
\usepackage{setspace}
\usepackage{footnote}
\usepackage{wasysym}
% Seitenstil definieren
\pagestyle{fancy}
\fancyhead{}
\fancyfoot{}
\rhead{\leftmark}
\cfoot{\thepage}
\renewcommand{\headrulewidth}{0pt}
\renewcommand{\footrulewidth}{0.4pt}

\usepackage[bottom]{footmisc}
%\usepackage{footmisc}
\setlength{\footnotemargin}{2mm} % Einr�cken der Fu�note
\setlength{\footnotesep}{12pt}% Abstand zwischen den Fu�noten 
\setlength{\skip\footins}{22pt}% Abstand Zwischen Haupttext und Fu�noten

%-------------------------------
\usepackage{calc}

% Paket f�r Zeichnungen
\usepackage{tikz}
\usepackage{tikz-timing}
\usepackage{vhistory}
%

\usetikztiminglibrary[new={char=Q,reset char=R}]{counters}
\usetikztiminglibrary{arrows}
\usetikzlibrary{shapes,
								arrows,
								calc,
								automata,
								positioning,
								mindmap,
								fit,
								trees,
								shadows,
								decorations,
								scopes,
								matrix,
								chains,
								shapes.misc,% wg. rounded rectangle
  							shapes.arrows,%
  							decorations.pathmorphing}



%-------------------------------
\usepackage[plainpages=false]{hyperref}
\hypersetup{colorlinks%Rot im inhaltsverzeichniss entfernt"'
,linkcolor=black
,filecolor=blue
,urlcolor=blue
,citecolor=blue}

%\pdfoptionpdfminorversion=6
%\pdfminorversion=7  %akzeptiert pdf in version 1.7
\setcounter{tocdepth}{3} %%f�r subsubsection!!!
\setcounter{secnumdepth}{3} 
%\usepackage{watermark}
\definecolor{light}{gray}{.50}



%%%%%%%%% Neue Kommandos %%%%%%%%%%%%%
\let\abk\nomenclature % Abk�rzung shortcut
\newcommand{
\changefont}[3]{\fontfamily{#1} \fontseries{#2} \fontshape{#3} \selectfont} % Setzen der Schriftart



%%%%%%%%%%%%%% zus�tzliche unit-Spalte %%%%%%%%%%%%%%%%
\newcommand{\nomunit}[1]{%
\renewcommand{\nomentryend}{\hspace{2em}\hspace*{\fill}#1}}


%%%%%%%%%%%%%% longtable an Stelle der Liste %%%%%%%%%%
\makeatletter
\def\@@@nomenclature[#1]#2#3{%
   \def\@tempa{#2}\def\@tempb{#3}%
   \protected@write\@nomenclaturefile{}%
      {\string\nomenclatureentry{#1\nom@verb\@tempa @{\nom@verb\@tempa}&%
         \begingroup\nom@verb\@tempb\protect\nomeqref{\theequation}%
            |nompageref}{\thepage}}%
   \endgroup
   \@esphack}
\def\thenomenclature{%
   \@ifundefined{chapter}{\section*}{\chapter*}{\nomname}%
   \nompreamble
   \begin{longtable}[l]{@{}ll@{}}}
\def\endthenomenclature{%
\end{longtable}
\nompostamble}
\makeatother


%%% Listings
\definecolor{LinkColor}{rgb}{0,0,0.5}
\definecolor{ListingBackground}{rgb}{0.9,0.9,0.9}
\definecolor{dunkelgrau}{rgb}{0.8,0.8,0.8}
\definecolor{lightblue}{rgb}{0.6132,0.7382,0.8554}
\definecolor{lightwhite}{rgb}{1,1,1}
\definecolor{schwarz}{rgb}{0,0,0}
\definecolor{weiss}{rgb}{1,1,1}
\definecolor{codegreen}{rgb}{0,0.5,0}
\definecolor{codegray}{rgb}{0.0,0.5,0.5}
\definecolor{codepurple}{rgb}{0.5,0,0.5}
\definecolor{backcolour}{rgb}{0.95,0.95,0.92}


\lstloadlanguages{C} % TeX sprache laden, notwendig wegen option 'savemem'
\lstset{%
	language=C,     				% Sprache des Quellcodes ist TeX
commentstyle=\color{codegreen},
    keywordstyle=\color{blue}\bfseries,
    numberstyle=\tiny\color{codegray},
    stringstyle=\color{codepurple},
	numbers=left,            % Zelennummern links
	stepnumber=1,            % Jede Zeile nummerieren.
	numbersep=5pt,           % 5pt Abstand zum Quellcode
	numberstyle=\tiny,       % Zeichengr�sse 'tiny' f�r die Nummern.
	breaklines=true,         % Zeilen umbrechen wenn notwendig.
	breakautoindent=true,    % Nach dem Zeilenumbruch Zeile einr�cken.
	postbreak=\space,        % Bei Leerzeichen umbrechen.
	tabsize=2,               % Tabulatorgr�sse 2
	basicstyle=\ttfamily\footnotesize, % Nichtproportionale Schrift, klein f�r den Quellcode
	showspaces=false,        % Leerzeichen nicht anzeigen.
	showstringspaces=false,  % Leerzeichen auch in Strings ('') nicht anzeigen.
	extendedchars=true,      % Alle Zeichen vom Latin1 Zeichensatz anzeigen.
	captionpos=b, %Caption unten
	backgroundcolor=\color{weiss}} % Hintergrundfarbe des Quellcodes setzen.


%%%%%%%%%%%%%%%%%%%%%%%%%%%%%%%%%%%%%%%%%%

\lstdefinelanguage{Assembler}%
   {morekeywords={BL,LDR,MOV,bx,sub,stmfd,ldr,str,mrs,msr,ldmia},%
    morekeywords=[2]{.global,.extern,.macro,.endm},%
    alsoletter={.,0,1,2,3,4,5,6,7,8,9},%
    alsodigit={?},%
    sensitive=false,%
    morestring=[b]",%
    morecomment=[s]{/*}{*/},%
    morecomment=[l]@,%
    morecomment=[l]//,%
   }[keywords,comments,strings]

%%%%%%%%% Erstellung der Verzeichnisse %%%%%%%%%%%%%
\makenomenclature
%\makeglossaries
%\makeindex

%%%%%%%%% Seitenlayout %%%%%%%%%%%%%
\setlength{\parindent}{0pt} % Zeileneinzug bei Absatz
\linespread{1.3} % hor. Zeilenabstand
%\geometry{a4paper, top=40mm, left=35mm, right=25mm, bottom=40mm, headsep=10mm, footskip=10mm} % Randabst�nde kl�ren

\geometry{a4paper, top=40mm, left=30mm, right=30mm, bottom=40mm, headsep=10mm, footskip=10mm} % Randabst�nde kl�ren
\setlength{\headheight}{15pt}

