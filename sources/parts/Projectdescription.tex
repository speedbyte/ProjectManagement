\chapter{Task description}The task is to make a graphical user interface to display the values of the sensor that are received by the oscilloscope via the UDP protocol.


\chapter{Development Environment} 
\textbf{Coding Language}: The project is developed on Java Version 7. To download the version 7, simply do yum install java-1.7.0-openjdk. For the sake of simplicity, the Java Version 7 should also be available on Windows as well.\\\\
\textbf{Versioning}: The software is versioned via GIT and is secured on the remote server. The sensor oscilloscope can be cloned via git@atreus.informatik.uni-tuebingen.de:agrawal/sensoroscilloscope.git. To test the udp connection, a set of python scripts has been written and they should be also cloned by running the command git clone git@atreus.informatik.uni-tuebingen.de:agrawal/python\_scripts.git\\

\chapter{Executing the Software}
When the software is executed, the user needs to select the radio button UDP and press the start button. As soon as the button is pressed, a server is started and starts listening at the port number \textbf{9876} on the local host.\\
To test if the server was successfully started, run the command nmap localhost -p 0-10000. This will list all the open ports.\\
It is required to add the panels in the oscilloscope before sending test udpdata to the oscilloscope. If the entry of the port at 9876 is available, then run the python script udpsend.py. 



