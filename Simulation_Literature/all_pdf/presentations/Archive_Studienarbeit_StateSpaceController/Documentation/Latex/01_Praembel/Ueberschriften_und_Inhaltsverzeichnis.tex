% Ueberschriften_und_Inhaltsverzeichnis
% *************************************


% Nummerierung der �berschriften bis zur Ebene 5
\setcounter{secnumdepth}{5}
% oder eben bis zur vierten Ebene...
%\setcounter{secnumdepth}{4}

% Mit diesem Befehl wird die die Tiefe des Inhaltsverzeichnisses ausgew�hlt.
\setcounter{tocdepth}{4}


% �berschriften der Ebene 4 und 5 (paragraph und subparagraph) abgesetzt formatieren:
% Der nachfolgende Text beginnt also in einer neuen Zeile - ist standardm��ig nicht der Fall
% Quelle(n):
% 1. http://www.zdv.uni-tuebingen.de/static/skripte/tech/tech_skript_f.pdf
% 2. LaTeX-Begleiter, z. B. auf Seite 32 (2. Auflage)
% ----------------------------------
\makeatletter
\renewcommand\paragraph{\@startsection
{paragraph}{4}{0em}                                % {Name}{Ebene}{Einzug}
{\baselineskip}{.2\baselineskip}                   % {vor-Abstand}{nach-Abstand}
{\normalsize\bfseries}}                            % {Layout}
\makeatother
% ----------------------------------
\makeatletter
\renewcommand\subparagraph{\@startsection
{subparagraph}{5}{0em}                              % {Name}{Ebene}{Einzug}
{\baselineskip}{.1\baselineskip}                    % {vor-Abstand}{nach-Abstand}
{\normalsize\bfseries}}                             % {Layout}
\makeatother
% ---------------------------------- 