% Standardpakete
% **************

% sprachspezifische Anpassungen
% -----------------------------

\usepackage[english,ngerman]{babel}

% Seitenlayout
% ------------

\usepackage[a4paper, driver=pdftex]{geometry}
\geometry
{% siehe geometry.pdf auf Seite 4 (Figure 1)
	left=3cm,
	right=3cm,
	bottom=3cm,
	top=3cm,
	%showframe=true, %Rahmen anzeigen lassen (auf erster Seite)
	headheight=2cm,
	headsep=0.5cm,
	footskip=1cm,
	% zus�tzlicher Rand f�r Bindung
	bindingoffset=0cm
}

% Kodierung und Schriften
% -----------------------

% Zeichenkodierung
\usepackage[latin1]{inputenc}
\usepackage[T1]{fontenc}

% Das Paket `helvet' legt `Helvetica' als serifenlose Schrift fest
% Das Paket `courier' legt `Courier' als Schreibmaschinenschrift fest.
\usepackage[scaled=1]{helvet}
\usepackage{courier}

% alternativ - �hnlich der LaTeX-StabdardschrifrStandardschrift Computer Modern (unterst�tzt Zeichen mit Akzenten etc. besser)
%\usepackage{lmodern}


% You may notice that LaTeX�s standard font (Computer Modern) is only available OT1-encoded (TeX�s standard). This makes TeX spit out ugly bitmapped fonts with T1 encoding by default.
% To overcome this, you could load the alternative Latin Modern Fonts (LM) which have been created as an extension of the CM for (not only) all sort of accented characters.
% Hence, try out to have TeX nicely typeset accented characters.
% http://www.win.ua.ac.be/~nschloe/content/top-10-latex-modules

% kleine typografische Anpassungen - nicht wichtig, sieht aber besser aus
% -----------------------------------------------------------------------

\usepackage{microtype}

% Physikalische Einheiten einfach verwenden
% Einheiten werden im Text- und Mathematikmodus aufrecht gesetzt.
% Im Textmodus wird die Formatierung des restlichen Text �bernommen (fett etc.)
\usepackage{units}

%Beispiele:
%\unit[1]{m}
%\unitfrac[1]{m}{s}
%\nicefrac{m}{s}

% Mathe
% -----

% verbessertes Mathezeugs

% This package defines commands to access bold math symbols. The basic command
% is \bm which may be used to make the math expression in its argument be typeset
% using bold fonts. The syntax of \bm is: %\bm{math expression}
\usepackage{amsmath, amssymb, bm}

% Erweitert amsmath und behebt einige Bugs
\usepackage[fixamsmath,disallowspaces]{mathtools}

% Warnt bei Benutzung von Befehlen die mit amsmath inkompatibel sind.
\usepackage[all,warning]{onlyamsmath}

% aufrechte griechische Buchstaben: \upmu etc.
\usepackage{upgreek}

% Zitate - This package provides advanced facilities for inline and display quotations
\usepackage[]{csquotes}

% Farben verwenden
\usepackage{xcolor}

% bessere Unterst�tzung von Gleitumgebungen
\usepackage{float}
\floatplacement{figure}{H} %wenn nix anderes dran steht, dass wird das Objekt HIER platziert

% Beschriftung von Tabellen, Abbildungen etc. beeinflussen, am besten Doku dazu lesen:
% http://www.math.ntnu.no/~berland/latex/docs/caption.pdf
% nach float, rotating und subfigure!
% normal                provides `normal' captions, this is the default
% hang or isu           provides captions with hanging indention
% center                provides captions where each line is centered
% centerlast            provides captions where the last line of the
%                       paragraph is centered
% nooneline             if a caption ts on one line on the page,
%                       it will be centered. If you don't like this
%                       behaviour, just select this option.
% scriptsize...Large    sets the font size of the captions
% up, it, sl, sc,
% md, bf, rm, sf,
% or tt                 sets the font attribute of the caption labels.
% ruled                 supports ruled floats of the float package, see
%                       section 1.1 for details
\usepackage[hang,small,bf]{caption}
    \renewcommand{\captionfont}{\sffamily}
    \renewcommand{\captionlabelfont}{\sffamily\bfseries}



% Einbindung von Grafiken (pdf, png, jpg)
% siehe LaTeX-Begleiter, Auflage 2 ab Seite 636.
\usepackage[pdftex]{graphicx}

% Es stellt das Symbol der europ�ischen W�hrung, des Euro, in LaTeX zur Verf�gung.
% Die Spezifikation stammt aus dem c't-Magazin 11/98, Seite 211, und aus der Encyclopaedia Britannica, Book of the Year 2002.
% http://www.theiling.de/eurosym.html
% http://www.ctan.org/tex-archive/fonts/eurosym/
\usepackage{eurosym}
\DeclareInputText{128}{\euro} % ANSI code for euro: � \usepackage{eurosym}

% Standardpfad f�r Grafiken definieren, siehe LaTeX-Begleiter auf Seite 642 ff.
\graphicspath{{./03_Grafiken/}}

