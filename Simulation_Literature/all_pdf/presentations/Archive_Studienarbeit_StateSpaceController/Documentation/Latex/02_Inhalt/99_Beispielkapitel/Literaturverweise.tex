% Literaturverweise
% *****************

\subsection[Literaturverweise]{Literaturverweise}

Interessant ist auch~\cite{bib:Quelle2} und \cite{bib:Quelle1} -- so sieht's aus. Langsam wird es sp�t\ldots ich  gehe ins Bett. 

Zitieren will gelernt sein -- die TU~Berlin \cite{bib:www:TUBerlin} gibt folgende Ratschl�ge:

\BZitat{%
Ein Zitat muss exakt den Text der Referenz enthalten, d.h. Zeichen f�r Zeichen �bertragen werden. Fehler m�ssen ebenfalls �bernommen werden, k�nnen jedoch durch ein nachgestelltes [sic!] oder [!] als Fehler des zitierten Textes kenntlich gemacht werden. Auslassungen bzw. K�rzungen werden durch [...] angezeigt und d�rfen nicht sinnver�ndernd sein.
Zu jedem Zitat geh�rt eine eindeutig zugeordnete Seitenangabe der Fundstelle, meist am Ende des Zitats. Dabei ist entweder die eine, oder bei l�ngeren Zitaten ggf. die Anfangs- und die Endseite der zitierten Stelle anzugeben. Angaben wie f. oder ff. sind heute nicht mehr �blich und lassen den Verdacht entstehen, dass ungenau gearbeitet wurde oder die zitierte Quelle gar nicht vorgelegen hat.
Zitate bis drei Zeilen L�nge werden im Flie�text untergebracht und in Anf�hrungszeichen gesetzt. Die Verwendung von Anf�hrungszeichen ist dem echten Zitat vorbehalten. 'Uneigentliche', umgangssprachliche oder selbstkonstruierte Begriffe z.B. werden also nicht in normale, sondern in einfache Anf�hrungszeichen gesetzt oder anders ausgezeichnet (z.B. durch Kursivsetzung). Enth�lt ein Zitat selbst ein Zitat, also Anf�hrungszeichen, so werden statt diesen ebenfalls einfache Anf�hrungszeichen gesetzt. Ist ein Zitat l�nger als drei Zeilen, sollte es als einger�ckter Absatz mit kleinerer Schrift dargestellt werden. Die Anf�hrungszeichen entfallen dabei, die Referenz mit Seitenangabe ist jedoch zu nennen.
Es sollte immer aus erster Hand zitiert werden. Nur wenn eine Quelle nicht oder nur mit unverh�ltnism��ig hohem Aufwand zu beschaffen ist, darf das Zitat einer anderen Quelle entnommen werden. Dabei ist zuerst die Originalquelle zu nennen, dann 'zit. in' und schlie�lich die Quelle, in der das Zitat vorgefunden wurde.
}