% Eigene Befehle
% **************

\subsection[Eigene Befehle]{Eigene Befehle}

    \subsubsection{myComment}

        Nachfolgend wird der Befehl \texttt{myComment} aufgerufen:

        \myComment{Hmm - diesen Abschnitt sollte ich nochmal �berarbeiten -- wirkt ein wenig sinnlos.}

    \subsubsection{Abk�rzungen und oft verwendete Begriffe}

        \begin{description}
            \item[Kurzbefehl: zB] Es ist \textbf{\zB} wichtig, diese Dinge zu kl�ren.
            \item[Kurzbefehl: ua] Es ist \textbf{\ua} wichtig, ihr zu gratulieren.
            \item[Kurzbefehl: usw] Es ist wichtig, gewollt, angemessen \textbf{\usw}
            \item[Kurzbefehl: Dh] Ich glaube, dass das so nicht richtig ist, \textbf{\Dh} Du l�gst!
            \item[Begriff: rcp] \rcp ist ganz toll. Auch am Ende eines Satzes \rcp.
        \end{description}

    \subsubsection{Uneigentliche, umgangssprachliche oder selbstkonstruierte Begriffe}

    Ich habe einen Begriff \uusB{gegoogelt} -- die Verwendung von normalen Anf�hrungszeichen ist dem echten Zitat vorbehalten
    
    \subsubsection{Sonstiges}

    Grad-Zeichen f�r Winkel- und Temperaturangaben. Bei Winkelangaben folgt das Gradzeichen direkt der Zahl: \myWinkel{30}. Im Gegensatz zu Temperaturangaben: \myTemp{20}. Das Ganze in einer Mathematik-Umgebung:
    
    \begin{align}
        \theta & = \myTemp{20}\\
        \phi   & = \myWinkel{30}
    \end{align}
    
    Als Bruch: \myTempFrac{20}{s} \bzw \myTempCarf{20}{s}.

    \subsubsection{Zitate}

        Blockzitat (deutsch):

        \BZitat{\blindtext}

        Blockzitat (englisch):

        \EngBZitat{LaTeX is intended to provide a high-level language that accesses the power of TeX. LaTeX essentially comprises a collection of TeX macros and a program to process LaTeX documents. Because the TeX formatting commands are very low-level, it is usually much simpler for end-users to use LaTeX.}

        Und schlie�lich ein Zitat in Anf�hrungszeichen \Zitat{Tolles Zitat in Anf�hrungszeichen}, so -- das m�sste klappen.            