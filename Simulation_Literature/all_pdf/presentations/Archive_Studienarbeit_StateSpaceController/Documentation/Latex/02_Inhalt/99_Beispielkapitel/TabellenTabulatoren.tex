% TabellenTabulatoren
% *******************

\subsection{Tabellen und Tabulatoren}

Tabulatorumgebung:

\begin{tabbing}
    �berschrift1 \hspace{1cm} \= �berschrift2 \hspace{1cm} \= �berschrift3\\
    1 \> lala \> lulu\\
    1 \> lala \> lulu\\
    1 \> lala \> lulu\\
\end{tabbing}

Angabe einer Musterzeile, f�r die Tabulator-Positionen (\texttt{kill}-Befehl):

\begin{tabbing}
    Musterbreite1 \hspace{1cm} \= Musterbreite2 \hspace{1cm} \= Musterbreite3  \= \kill
    �berschrift1 \hspace{1cm} \> �berschrift2 \hspace{1cm} \> �berschrift3\\
    1 \> lala \> lulu\\
    1 \> lala \> lulu\\
    1 \> lala \> lulu\\
\end{tabbing}

Horizontaler Abstand links:

\begin{tabbing}
    \hspace*{2cm} \= Musterbreite1 \hspace{1cm} \= Musterbreite2 \hspace{1cm} \= Musterbreite3  \= \+ \kill
    �berschrift1 \hspace{1cm} \> �berschrift2 \hspace{1cm} \> �berschrift3\\
    1 \> lala \> lulu\\
    1 \> lala \> lulu\\
    1 \> lala \> lulu\\
\end{tabbing}

% siehe LaTeX Praxisbuch, ISBN: 3772361099, Seite 93 bzw. Listing 3.3
% \hspace* -> f�gt auch Leerraum am Anfang einer Zeile ein
% \+ bzw. \- der linke Rand wird einen Tabstop weiter bzw. zur�ck gesetzt, kann mehrmals verwendet werden