% Einheiten
% *********

\subsection{Einheiten}

Einheiten mit dem \texttt{units}-Paket:

\begin{itemize}
    \item Textmodus: serifenlos und fett {\sffamily\bfseries\unit{m}}
    \item Mathemodus: serifenlos und fett {\sffamily\bfseries$\unit{m}$} $\to$ Im Mathematikmodus werden Textattribute ignoriert (au�er der Schriftgr��e.
    \item Textmodus: serifenlos und kursiv {\sffamily\itshape\unit[1]{m}}
    \item Mathemodus: serifenlos und kursiv {\sffamily\itshape$\unit[1]{m}$}
    \item Textmodus: serifenlos und kursiv {\sffamily\bfseries\unitfrac{m}{s}}
    \item Mathemodus: serifenlos und kursiv {\sffamily\bfseries$\unitfrac{m}{s}$}  
\end{itemize}

Beispiel mit komplexeren Ausdr�cken:

Die Masse betrug $\unit[2\cdot 10^{-3}]{kg}$ (Mathemodus) oder \unit[2$\cdot$ 10\textsuperscript{-3}]{kg} (Textmodus) -- also nimmt man lieber den Mathemodus!

Weiteres Beispiel: $\unit[(2\pm 0,2)\cdot 10^{-3}]{kg}$ oder $\unitfrac[(2\pm 0,2)\cdot 10^{-3}]{kg}{m^3}$. Man kann auch nur nur \emph{sch�ne Br�che} verwenden: \nicefrac[\bfseries]{Z�hler}{Nenner}. Der \texttt{nicefrac}-Befehl hat drei Parameter. Der optionale Parameter ist f�r ein so genanntes \texttt{fontcommand} und kann Formatierungen oder einen mathematischen Schriftsatz beinhalten. Beispiele:

\begin{itemize}
    \item {\bfseries\itshape\nicefrac{1}{2}}
    \item {\nicefrac[\texttt]{1}{2}}
    \item {$\nicefrac[\mathcal]{A}{B}$}
\end{itemize}


