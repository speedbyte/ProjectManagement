\chapter{Aims and Objectives}

The aim of this dissertation is to provide a simulation architecture that can be
used as prototype development platform for a distributed visual movement
detection and control of a quadrocopter. Furthermore the characteristics of the
distributed image processing and movement detection have to be analyzed in
relation to the variation of configurations and critically assessed.

One essential objective is that the configuration of the simulating components
provides the option to simulate a range of hardware components which are not
purchased until now. By way of example the simulation of the on-Board camera has
to provide options of configuration for the resolution, color intensity, blur and
so far. The simulation of the communication between \UAV and host also has to
provide a variation of transmission rate and further behaviour which could affect
the visual movement detection at the base station. The efficiency in relation
with the quality of function is an important indicator for the success and acceptance of
the distributed movement detection approach. So it is important to get an insight
to the possible characteristics of the simulated components with the result to
find a way which satisfies the efficiency and quality aspects.

Another important objective is that the interfaces between the simulating
components are clearly specified and allow a way of modular exchangeability of
simulation components with the real objects.\newpage This purpose has to allow a
more precise investigation of the behaviour of the real hardware related components
and the option to test software for the \UAV target, like the On-Board control
algorithm, at the base station.

The realization of the simulation therefore has to provide an encapsulated and
flexible architecture and has to simulate behaviour like delays and jitters for
simulated components. Thereby the simulation has to adjust a real-time-behaviour
in the complete simulated environment and to allow so measurements and prediction
of feasibilities with the simulated configuration.
