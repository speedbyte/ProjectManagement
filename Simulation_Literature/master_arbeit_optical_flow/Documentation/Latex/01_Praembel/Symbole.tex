% Symbole
% *******


% alle Symbole siehe 98_Sonstiges/symbols-a4.pdf

% -------------------------------------------------------
% textcomp-Paket
% Das Erg�nzungspaket textcomp stellt verschiedene Symbole f�r den Textmodus
% bereit (mathcomp analog f�r Mathematikmodus).
% Es baut auf den EC-Schriften auf (hier: cmr).
% Verwendet man andere Schriften, so stehen nicht alle Symbole zur Verf�gung.

% Mit der Paket-Option force kann man erwzingen, dass Zeichen immer ausgegeben werden,
% auch wenn die Gefahr besteht, dass das Symbol in der Schriftart nicht verf�gbar ist
% -> man bekommt dann evtl. ein gef�lltes Viereck

% Oder besser:  [force,almostfull].  Im Gegensatz zu [force] alleine
% werden dann wenigstens diejenigen Zeichen automatisch durch CM
% ersetzt, die in nicht-CM-Fonts so gut wie immer fehlen, n�mlich
% \t und \textcircled.

% Symbole:      98_Sonstiges/textcomp.pdf
% Hinweise:     98_Sonstiges/textcomptst.pdf

\usepackage{textcomp}
\usepackage{mathcomp}
% Textmodus:        \textleaf
% Mathematikmodus:  $\tcleaf$
% -------------------------------------------------------

% -------------------------------------------------------
% marvosym-Paket - vertr�gt sich nicht mit dem textcomp-Paket: entweder oder

% Weitere Symbole

% Doku: 98_Sonstiges/marvodoc.pdf

% Paket �berschreibt den Mathematik-Befehl "\Rightarrow" -> nachfolgend ein Workarround, um den urspr�nglichen Befehl zu "retten".

% \let\OldRightarrow\Rightarrow % alte Bedeutung retten
% \usepackage{marvosym} %\Rightarrow wird umdefiniert
% \let\Rightarrow\OldRightarrow % alte Bedeutung wird restauriert.

% \let: Low-Level-TeX-Anweisung um einem Makro die Definition eines anderen Makros zuzuweisen. 
%--------------------------------------------------------

% Weitere Pakete: wasysym, pifont, latexsym