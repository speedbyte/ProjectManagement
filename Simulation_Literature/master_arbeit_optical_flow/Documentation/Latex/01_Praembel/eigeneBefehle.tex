% eigeneBefehle
% *************


% \newcommand{\Name}[Anzahl]{Definition}
% Neuen Befehl definieren
% --------------------------------------

% mit phi wird so das normale phi-Symbol gesetzt
% ----------------------------------------------

\renewcommand{\phi}{\varphi}


% Setzt auf das optional- und color-Paket auf - siehe "Standardpakete"
% Ausschalten bei Spezialpakete
% --------------------------------------------------------------------

\newcommand{\myComment}[1]
{
    \opt{myComment}
    {
        \begin{quote}
            \color{red}\textbf{Bearbeitungshinweis:} #1
        \end{quote}
    }
}

% �nderungen in Bezug auf Verweise
% --------------------------------

% Anmerkung:
% "Muss" mit ref{ enden damit in WinEdt das Macro Ref.edt noch funktioniert und
% die Liste mit Labels aufpoppt wenn man den Befehl verwendet

% baut auf dem Paket varioref auf
% siehe LaTeX-Begleiter ab Seite 72 (2. Auflage)

\newcommand{\myfigpageref}[1]{\figurename~\ref{#1}~\vpageref{#1}}
\newcommand{\myfigref}[1]{\figurename~\ref{#1}}

\newcommand{\myeqpageref}[1]{Gleichung~\ref{#1}~\vpageref{#1}}
\newcommand{\myeqref}[1]{(equation~\ref{#1})}

\newcommand{\myeqRangeref}[2]{Gleichungen~\vrefrange{#1}{#2}}

% siehe LaTeX-Begleiter auf Seite 75 (2. Auflage)

% Befehle f�r Abk�rzungen
% -----------------------

\newcommand{\zB}{z.\,B.\xspace}
\newcommand{\bzw}{bzw.\xspace}
\newcommand{\ua}{u.\,a.\xspace}
\newcommand{\usw}{usw.\@\xspace} %siehe LaTeX-Begleiter auf Seite 87, 2. Auflage
\newcommand{\Dh}{d.\,h.,\xspace} % \dh ist als Befehl schon belegt (keine Ahnung durch was), daher \Dh


%Gradzeichen - ben�tigt die Pakete: textcomp, amsmath und units
\newcommand{\myWinkel}[1]{\text{#1\textdegree}}
\newcommand{\myTemp}[1]{\unit[#1]{\text{\textdegree}C}}
\newcommand{\myTempFrac}[2]{\unitfrac[#1]{\text{\textdegree}C}{#2}}
\newcommand{\myTempCarf}[2]{\unitfrac[#1]{#2}{\text{\textdegree}C}}

% Zitierten
% ---------

% Anf�hrungszeichen f�r w�rtliche Zitate
\newcommand{\Zitat}[1]{
\glqq#1\grqq\xspace
}


% Blockzitat
\newcommand{\BZitat}[1]{
\begin{quote}
    \small
    #1
\end{quote}
}

% englisches Blockzitat
\newcommand{\EngBZitat}[1]{
\begin{quote}
    \small
    \begin{otherlanguage}{english}
        #1
    \end{otherlanguage}
\end{quote}
}

% --------------------------------------------------------------------

% Formatierung f�r uneigentliche, umgangssprachliche oder selbstkonstruierte Begriffe (-> uusB)
% Denn: Die Verwendung von Anf�hrungszeichen ist dem echten Zitat vorbehalten!
% http://www.ak.tu-berlin.de/menue/forschung/zitieren_und_verweisen/
\newcommand{\uusB}[1]{%
'#1'\xspace
}


\makeatletter
\newif\if@borderstar
\def\bordermatrix{\@ifnextchar*{%
\@borderstartrue\@bordermatrix@i}{\@borderstarfalse\@bordermatrix@i*}%
}
\def\@bordermatrix@i*{\@ifnextchar[{\@bordermatrix@ii}{\@bordermatrix@ii[()]}}
\def\@bordermatrix@ii[#1]#2{%
\begingroup
\m@th\@tempdima8.75\p@\setbox\z@\vbox{%
\def\cr{\crcr\noalign{\kern 2\p@\global\let\cr\endline }}%
\ialign {$##$\hfil\kern 2\p@\kern\@tempdima & \thinspace %
\hfil $##$\hfil && \quad\hfil $##$\hfil\crcr\omit\strut %
\hfil\crcr\noalign{\kern -\baselineskip}#2\crcr\omit %
\strut\cr}}%
\setbox\tw@\vbox{\unvcopy\z@\global\setbox\@ne\lastbox}%
\setbox\tw@\hbox{\unhbox\@ne\unskip\global\setbox\@ne\lastbox}%
\setbox\tw@\hbox{%
$\kern\wd\@ne\kern -\@tempdima\left\@firstoftwo#1%
\if@borderstar\kern2pt\else\kern -\wd\@ne\fi%
\global\setbox\@ne\vbox{\box\@ne\if@borderstar\else\kern 2\p@\fi}%
\vcenter{\if@borderstar\else\kern -\ht\@ne\fi%
\unvbox\z@\kern-\if@borderstar2\fi\baselineskip}%
\if@borderstar\kern-2\@tempdima\kern2\p@\else\,\fi\right\@secondoftwo#1 $%
}\null \;\vbox{\kern\ht\@ne\box\tw@}%
\endgroup
}
\makeatother