% Beispielkapitel
% ***************

\section[Kapitel der h�chsten Hierarchie (TOC-Eintrag)]{Kapitel der h�chsten (Dokument-Eintrag)}

Im vorliegenden Beispielkapitel gehe ich auf die einzelnen Features ein. Als F�lltext verwende ich einen so genannten \emph{Blindtext}:

\blindtext

Hier noch kurz die Verwendung des \euro-Zeichens:  Das Buch kostet \EUR{10}. Es geht auch fett: \textbf{Das Buch kostet \EUR{10}.}

% Grafik und Referenzen
% *********************

\subsection[Grafiken und Referenzen]{Grafiken und Referenzen bzw. Verweise}

In diesem (Unter-)Kapitel werden Grafiken eingebunden und Verweise darauf erzeugt. Wir beginnen mit der leichtesten �bung -- der Einbindung eines Fotos mit Bildunterschrift:

\begin{figure}
    \centering
        \includegraphics[width=0.9\textwidth]{03_Grafiken/Foto.jpg}

    % In der Praembel unter Standardpakete ist das Standardverzeichnis f�r Grafiken
    % definiert - daher reicht hier der Dateiname ohne (relativer oder absoluter) Pfadangabe.

    \caption[Foto einer Asynchronmaschine (LOF-Eintrag)]{Foto einer Asynchronmaschine (Dokument-Eintrag)}
    \label{fig:Foto.jpg}
\end{figure}

\blindtext

Mit Hilfe eines selbst definierten Befehles erzeuge ich nun einen Verweis auf das Foto: In \myfigref{fig:Foto.jpg} sehen Sie einen Motor. 

% Textformatierungen
% ******************

\subsection[Textformatierungen]{Textformatierungen}


Die Standard Text Schalter erzeugen
\textbf{fett},
\textit{kursiv (italic)},
\textsl{schr�ggestellt (slanted)},
\textsf{serifenlos (grotesk)},
\textsc{Kapit�lchen} und
\texttt{Schreibmaschinenschrift}.
Sowie beliebige Kombinationen derselben:
\textit{\textbf{fett kursiv}},
\textsl{\textbf{fett schr�g}},
\textsf{\textbf{fett serifenlos}},
\textsc{\textbf{Fette Kapit�lchen}}
und
\textsl{\textsf{serifenlos schr�g}}. Je nach Schrift sind jedoch nicht alle Kombinationen m�glich. In dem Fall bekommt man die Fehlermeldung `Some font shapes were not available, defaults substituted.' 

% Listen
% ******

% Weitere Infos zu Listen siehe
% - LaTeX-Begleiter ab S. 135
% - 98_Sonstiges\LaTeX-Dokumente\"LaTeX - Fortgeschrittene Anwendungen Fernuni Hagen.pdf"

% Listen anpassen
% ---------------

% itemize
\renewcommand\labelitemi{\textbullet}                       %Standard: \textbullet
\renewcommand\labelitemii{\normalfont\bfseries\textendash}  %Standard: \normalfont\bfseries\textdash
\renewcommand\labelitemiii{\textasteriskcentered}           %Standard: \textasteriskcentered -> *
\renewcommand\labelitemiv{\textperiodcentered}              %Standard: \textperiodcentered

% Beispiel f�r eine eigene itemize-Umgebung
\newenvironment{myItemize}
{\renewcommand\labelitemi{\normalfont\bfseries\textendash}\begin{itemize}}
{\end{itemize}}

% siehe Latex-Begleiter Seite 136 ff.
% Aufz�hlungen (enumerate-Umgebung) flexibler handhaben
\usepackage{enumerate}

% description
% �ndern der Formatierung des description-Labels
% siehe Latex-Begleiter auf Seite 139
\renewcommand\descriptionlabel[1]%
{\hspace{\labelsep}\textbf{#1}} %Standard: \hspace{\labelsep}\textbf{#1}


% Literaturverweise
% *****************

\subsection[Literaturverweise]{Literaturverweise}

Interessant ist auch~\cite{bib:Quelle2} und \cite{bib:Quelle1} -- so sieht's aus. Langsam wird es sp�t\ldots ich  gehe ins Bett. 

Zitieren will gelernt sein -- die TU~Berlin \cite{bib:www:TUBerlin} gibt folgende Ratschl�ge:

\BZitat{%
Ein Zitat muss exakt den Text der Referenz enthalten, d.h. Zeichen f�r Zeichen �bertragen werden. Fehler m�ssen ebenfalls �bernommen werden, k�nnen jedoch durch ein nachgestelltes [sic!] oder [!] als Fehler des zitierten Textes kenntlich gemacht werden. Auslassungen bzw. K�rzungen werden durch [...] angezeigt und d�rfen nicht sinnver�ndernd sein.
Zu jedem Zitat geh�rt eine eindeutig zugeordnete Seitenangabe der Fundstelle, meist am Ende des Zitats. Dabei ist entweder die eine, oder bei l�ngeren Zitaten ggf. die Anfangs- und die Endseite der zitierten Stelle anzugeben. Angaben wie f. oder ff. sind heute nicht mehr �blich und lassen den Verdacht entstehen, dass ungenau gearbeitet wurde oder die zitierte Quelle gar nicht vorgelegen hat.
Zitate bis drei Zeilen L�nge werden im Flie�text untergebracht und in Anf�hrungszeichen gesetzt. Die Verwendung von Anf�hrungszeichen ist dem echten Zitat vorbehalten. 'Uneigentliche', umgangssprachliche oder selbstkonstruierte Begriffe z.B. werden also nicht in normale, sondern in einfache Anf�hrungszeichen gesetzt oder anders ausgezeichnet (z.B. durch Kursivsetzung). Enth�lt ein Zitat selbst ein Zitat, also Anf�hrungszeichen, so werden statt diesen ebenfalls einfache Anf�hrungszeichen gesetzt. Ist ein Zitat l�nger als drei Zeilen, sollte es als einger�ckter Absatz mit kleinerer Schrift dargestellt werden. Die Anf�hrungszeichen entfallen dabei, die Referenz mit Seitenangabe ist jedoch zu nennen.
Es sollte immer aus erster Hand zitiert werden. Nur wenn eine Quelle nicht oder nur mit unverh�ltnism��ig hohem Aufwand zu beschaffen ist, darf das Zitat einer anderen Quelle entnommen werden. Dabei ist zuerst die Originalquelle zu nennen, dann 'zit. in' und schlie�lich die Quelle, in der das Zitat vorgefunden wurde.
}

% eigeneBefehle
% *************


% \newcommand{\Name}[Anzahl]{Definition}
% Neuen Befehl definieren
% --------------------------------------

% mit phi wird so das normale phi-Symbol gesetzt
% ----------------------------------------------

\renewcommand{\phi}{\varphi}


% Setzt auf das optional- und color-Paket auf - siehe "Standardpakete"
% Ausschalten bei Spezialpakete
% --------------------------------------------------------------------

\newcommand{\myComment}[1]
{
    \opt{myComment}
    {
        \begin{quote}
            \color{red}\textbf{Bearbeitungshinweis:} #1
        \end{quote}
    }
}

% �nderungen in Bezug auf Verweise
% --------------------------------

% Anmerkung:
% "Muss" mit ref{ enden damit in WinEdt das Macro Ref.edt noch funktioniert und
% die Liste mit Labels aufpoppt wenn man den Befehl verwendet

% baut auf dem Paket varioref auf
% siehe LaTeX-Begleiter ab Seite 72 (2. Auflage)

\newcommand{\myfigpageref}[1]{\figurename~\ref{#1}~\vpageref{#1}}
\newcommand{\myfigref}[1]{\figurename~\ref{#1}}

\newcommand{\myeqpageref}[1]{Gleichung~\ref{#1}~\vpageref{#1}}
\newcommand{\myeqref}[1]{(equation~\ref{#1})}

\newcommand{\myeqRangeref}[2]{Gleichungen~\vrefrange{#1}{#2}}

% siehe LaTeX-Begleiter auf Seite 75 (2. Auflage)

% Befehle f�r Abk�rzungen
% -----------------------

\newcommand{\zB}{z.\,B.\xspace}
\newcommand{\bzw}{bzw.\xspace}
\newcommand{\ua}{u.\,a.\xspace}
\newcommand{\usw}{usw.\@\xspace} %siehe LaTeX-Begleiter auf Seite 87, 2. Auflage
\newcommand{\Dh}{d.\,h.,\xspace} % \dh ist als Befehl schon belegt (keine Ahnung durch was), daher \Dh


%Gradzeichen - ben�tigt die Pakete: textcomp, amsmath und units
\newcommand{\myWinkel}[1]{\text{#1\textdegree}}
\newcommand{\myTemp}[1]{\unit[#1]{\text{\textdegree}C}}
\newcommand{\myTempFrac}[2]{\unitfrac[#1]{\text{\textdegree}C}{#2}}
\newcommand{\myTempCarf}[2]{\unitfrac[#1]{#2}{\text{\textdegree}C}}

% Zitierten
% ---------

% Anf�hrungszeichen f�r w�rtliche Zitate
\newcommand{\Zitat}[1]{
\glqq#1\grqq\xspace
}


% Blockzitat
\newcommand{\BZitat}[1]{
\begin{quote}
    \small
    #1
\end{quote}
}

% englisches Blockzitat
\newcommand{\EngBZitat}[1]{
\begin{quote}
    \small
    \begin{otherlanguage}{english}
        #1
    \end{otherlanguage}
\end{quote}
}

% --------------------------------------------------------------------

% Formatierung f�r uneigentliche, umgangssprachliche oder selbstkonstruierte Begriffe (-> uusB)
% Denn: Die Verwendung von Anf�hrungszeichen ist dem echten Zitat vorbehalten!
% http://www.ak.tu-berlin.de/menue/forschung/zitieren_und_verweisen/
\newcommand{\uusB}[1]{%
'#1'\xspace
}


\makeatletter
\newif\if@borderstar
\def\bordermatrix{\@ifnextchar*{%
\@borderstartrue\@bordermatrix@i}{\@borderstarfalse\@bordermatrix@i*}%
}
\def\@bordermatrix@i*{\@ifnextchar[{\@bordermatrix@ii}{\@bordermatrix@ii[()]}}
\def\@bordermatrix@ii[#1]#2{%
\begingroup
\m@th\@tempdima8.75\p@\setbox\z@\vbox{%
\def\cr{\crcr\noalign{\kern 2\p@\global\let\cr\endline }}%
\ialign {$##$\hfil\kern 2\p@\kern\@tempdima & \thinspace %
\hfil $##$\hfil && \quad\hfil $##$\hfil\crcr\omit\strut %
\hfil\crcr\noalign{\kern -\baselineskip}#2\crcr\omit %
\strut\cr}}%
\setbox\tw@\vbox{\unvcopy\z@\global\setbox\@ne\lastbox}%
\setbox\tw@\hbox{\unhbox\@ne\unskip\global\setbox\@ne\lastbox}%
\setbox\tw@\hbox{%
$\kern\wd\@ne\kern -\@tempdima\left\@firstoftwo#1%
\if@borderstar\kern2pt\else\kern -\wd\@ne\fi%
\global\setbox\@ne\vbox{\box\@ne\if@borderstar\else\kern 2\p@\fi}%
\vcenter{\if@borderstar\else\kern -\ht\@ne\fi%
\unvbox\z@\kern-\if@borderstar2\fi\baselineskip}%
\if@borderstar\kern-2\@tempdima\kern2\p@\else\,\fi\right\@secondoftwo#1 $%
}\null \;\vbox{\kern\ht\@ne\box\tw@}%
\endgroup
}
\makeatother

% TabellenTabulatoren
% *******************

\subsection{Tabellen und Tabulatoren}

Tabulatorumgebung:

\begin{tabbing}
    �berschrift1 \hspace{1cm} \= �berschrift2 \hspace{1cm} \= �berschrift3\\
    1 \> lala \> lulu\\
    1 \> lala \> lulu\\
    1 \> lala \> lulu\\
\end{tabbing}

Angabe einer Musterzeile, f�r die Tabulator-Positionen (\texttt{kill}-Befehl):

\begin{tabbing}
    Musterbreite1 \hspace{1cm} \= Musterbreite2 \hspace{1cm} \= Musterbreite3  \= \kill
    �berschrift1 \hspace{1cm} \> �berschrift2 \hspace{1cm} \> �berschrift3\\
    1 \> lala \> lulu\\
    1 \> lala \> lulu\\
    1 \> lala \> lulu\\
\end{tabbing}

Horizontaler Abstand links:

\begin{tabbing}
    \hspace*{2cm} \= Musterbreite1 \hspace{1cm} \= Musterbreite2 \hspace{1cm} \= Musterbreite3  \= \+ \kill
    �berschrift1 \hspace{1cm} \> �berschrift2 \hspace{1cm} \> �berschrift3\\
    1 \> lala \> lulu\\
    1 \> lala \> lulu\\
    1 \> lala \> lulu\\
\end{tabbing}

% siehe LaTeX Praxisbuch, ISBN: 3772361099, Seite 93 bzw. Listing 3.3
% \hspace* -> f�gt auch Leerraum am Anfang einer Zeile ein
% \+ bzw. \- der linke Rand wird einen Tabstop weiter bzw. zur�ck gesetzt, kann mehrmals verwendet werden

\input{02_Inhalt/99_Beispielkapitel/Sprachwechsel}

% Mathematik und Referenzen
% *************************

\subsection{Mathematik und Referenzen}

\begin{equation}
a^2+b^2=c^2
\label{eq:lala}
\end{equation}

\myeqref{eq:lala} ist nicht so schwer.

\clearpage

\begin{equation}
a^2+b^2=c^2
\label{eq:lala2}
\end{equation} 

So, jetzt mal eine mehrzeilige Formel:

\begin{equation}
\label{eq:mehrzeilig1}
    \begin{split}
        A   &= \dfrac{B + B + B}{C\cdot C \cdot C} + \dfrac{B + B + B}{C\cdot C \cdot C} +\\
            &\hspace{1em} \dfrac{B + B + B}{C\cdot C \cdot C}
    \end{split}
\end{equation}


Alternativ (die Formelnummer ist hier nicht so sch�n zentriert wie im oberen Beispiel):

\begin{align}
\label{eq:mehrzeilig2}
        A   &= \dfrac{B + B + B}{C\cdot C \cdot C} + \dfrac{B + B + B}{C\cdot C \cdot C} + \nonumber\\
            &\phantom{=}\dfrac{B + B + B}{C\cdot C \cdot C}
\end{align}

Oder:

\begin{align}
\label{eq:mehrzeilig3}
        A   =   &\hspace{0.3em} \dfrac{B + B + B}{C\cdot C \cdot C} + \dfrac{B + B + B}{C\cdot C \cdot C} + \nonumber\\
                &\hspace{0.3em}\dfrac{B + B + B}{C\cdot C \cdot C}
\end{align}

% Listings einbinden
% ******************

\subsection{Listings einbinden}

\begin{figure}
    \lstinputlisting[caption={[Kurzfassung]Listing, das ganz toll ist.},label={listing:MatlabListing}]{06_Listings/MatlabListing.m}
\end{figure}

Siedhe Listing~\ref{listing:MatlabListing}

% Einheiten
% *********

\subsection{Einheiten}

Einheiten mit dem \texttt{units}-Paket:

\begin{itemize}
    \item Textmodus: serifenlos und fett {\sffamily\bfseries\unit{m}}
    \item Mathemodus: serifenlos und fett {\sffamily\bfseries$\unit{m}$} $\to$ Im Mathematikmodus werden Textattribute ignoriert (au�er der Schriftgr��e.
    \item Textmodus: serifenlos und kursiv {\sffamily\itshape\unit[1]{m}}
    \item Mathemodus: serifenlos und kursiv {\sffamily\itshape$\unit[1]{m}$}
    \item Textmodus: serifenlos und kursiv {\sffamily\bfseries\unitfrac{m}{s}}
    \item Mathemodus: serifenlos und kursiv {\sffamily\bfseries$\unitfrac{m}{s}$}  
\end{itemize}

Beispiel mit komplexeren Ausdr�cken:

Die Masse betrug $\unit[2\cdot 10^{-3}]{kg}$ (Mathemodus) oder \unit[2$\cdot$ 10\textsuperscript{-3}]{kg} (Textmodus) -- also nimmt man lieber den Mathemodus!

Weiteres Beispiel: $\unit[(2\pm 0,2)\cdot 10^{-3}]{kg}$ oder $\unitfrac[(2\pm 0,2)\cdot 10^{-3}]{kg}{m^3}$. Man kann auch nur nur \emph{sch�ne Br�che} verwenden: \nicefrac[\bfseries]{Z�hler}{Nenner}. Der \texttt{nicefrac}-Befehl hat drei Parameter. Der optionale Parameter ist f�r ein so genanntes \texttt{fontcommand} und kann Formatierungen oder einen mathematischen Schriftsatz beinhalten. Beispiele:

\begin{itemize}
    \item {\bfseries\itshape\nicefrac{1}{2}}
    \item {\nicefrac[\texttt]{1}{2}}
    \item {$\nicefrac[\mathcal]{A}{B}$}
\end{itemize}









