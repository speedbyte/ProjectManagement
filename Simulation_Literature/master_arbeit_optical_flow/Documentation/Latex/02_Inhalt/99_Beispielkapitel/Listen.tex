% Listen und Aufz�hlungen
% ***********************

\subsection[Listen und Aufz�hlungen]{Listen und Aufz�hlungen}

Nachfolgend ein paar Listen und Aufz�hlungen. Die erste Liste ist eines selbst definierte Liste:

    \begin{myItemize}
        \item Liste - aber mit Strichen!
        \item Liste - aber mit Strichen!
        \item Liste - aber mit Strichen!
    \end{myItemize}

So sieht eine Standard-Liste aus:

    \begin{itemize}
        \item Standard-Liste
        \item Standard-Liste
        \item Standard-Liste
    \end{itemize}

\blindtext

Eine sogenannte \texttt{description}-Umgebung:

    \begin{description}
        \item[soso] Soso beschreibt eigentlich gar nichts, es ist aber dennoch gut geeignet. Man k�nnte auch anderen sinnfreie Buchstabenkombinationen verwenden.
        \item[sosososo] Soso beschreibt eigentlich gar nichts, es ist aber dennoch gut geeignet. Man k�nnte auch anderen sinnfreie Buchstabenkombinationen verwenden.
        \item[soso] Soso beschreibt eigentlich gar nichts, es ist aber dennoch gut geeignet. Man k�nnte auch anderen sinnfreie Buchstabenkombinationen verwenden.
    \end{description}

Aufz�hlungen, die flexibel mit dem \texttt{enumerate}-Paket gestaltet werden k�nnen:

    \begin{enumerate}[{Element} a)]
        \item Aufz�hlungselement
        \item \textbf{referenziertes} Aufz�hlungselement\label{list:bla}
        \item Aufz�hlungselement
        \item Aufz�hlungselement
    \end{enumerate}

    \begin{enumerate}[{Element} A)]
        \item Aufz�hlungselement
        \item Aufz�hlungselement
        \item Aufz�hlungselement
    \end{enumerate}

Man kann auf auf einzelne Eintr�ge verweisen, so zum Beispiel auf das \textbf{referenzierte} Aufz�hlungselement~\ref{list:bla}. Eine weitere Aufz�hlung, die mit dem \texttt{enumerate}-Paket erfolgt:

    \begin{enumerate}[{A}-1]
        \item Aufz�hlungselement
        \item Aufz�hlungselement
        \item Aufz�hlungselement
        \item Aufz�hlungselement
    \end{enumerate}

Auch eine r�mische Nummerierung ist m�glich -- aber mir gef�llt das nicht:

    \begin{enumerate}[{Element} i)]
        \item Aufz�hlungselement
        \item Aufz�hlungselement
        \item Aufz�hlungselement
    \end{enumerate}

    \begin{enumerate}[{Element} I)]
        \item Aufz�hlungselement
        \item Aufz�hlungselement
        \item Aufz�hlungselement
    \end{enumerate}    