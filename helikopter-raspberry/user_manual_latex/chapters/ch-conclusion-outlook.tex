%--------------------------------------------
% Chapter: CONCLUSION AND OUTLOOK
%--------------------------------------------
\chapter{Conclusion and outlook}
\label{sec:conclusion}

%- - - - - - - - - - - - - - - - - - - - - - 
% Section: Achieved goals and results
%- - - - - - - - - - - - - - - - - - - - - - 
\section{Achieved project goals and results}
\label{sec:conclusion:results}

First of all a project plan for the two groups was set up and the needed hardware was chosen. Following this a plan of the mounting and wiring was drawn and two prototypes were build up.\\
To increase the flexibility of using various operating systems on the development computer an Ubuntu system on a virtual machine is used. Programming is done by Eclipse which also is installed in the virtual machine. Additionally cross compiling and the ability to debug is enabled. Due to the fact that the compiling runs on the development computer, the speed of compiling is dramatically increased. The remote debugging feature helps to track down and fix errors during the development process. The already set up development environment can be downloaded from the SVN repository.\\
To ensure real-time capability, a real time patch is applied to the Raspbian distribution of the Raspberry Pi.\\
By introducing of hierarchies in the software, the hardware dependency is separated to just one layer. This makes the software better portable. In case of hardware changes, only in the affected layer software has to be changed. With this in mind, it was split up in Low level driver, the hardware abstraction, signal processing and application layer.\\
Beginning with the programming an interface abstraction to the Low level drivers of the UART was developed and proofed. The also needed interface abstraction of the I2C driver was implemented by the second project group. Just some improvements were done by the authors of this document. The Graupner PPM decoding is enabled via a Kernel module which can be loaded directly to the Kernel of the system. This enables interrupting with a very low jitter. This is needed to provide accurate measuring of the PPM pulses of the remote control. Additionally an experimental time trigger is provided via a second Kernel module driver. This is used to provide accurate timing for the control loop. The patched and pre-configured operating system of the Raspberry Pi can also be downloaded from the SVN repository.\\
After successful testing of the low level drivers in conjunction with the interface abstraction, the development of the hardware abstraction layer was started. The implementation of the GPS driver and the Inertial measurement unit sensors was started and successfully finished. All of these modules provide an data interface for the next hierarchy. All the drivers are tested as single units to ensure proper working.\\
The last step of the implementation was the signal processing layer. On this hierarchy an reduced IMU-Filter and the Orientation fusion was implemented.\\
The sensor fusion covers the successful implementation of the complementary Filter and the Kalman Filter. Additionally a generic matrix library was programmed to enable Matrix operations for the Kalman Filter. This library can be used with matrices of various sizes, written in pure C-code. The sensor fusion provides the absolute orientation of the system. The orientation angles of roll, pitch and yaw are with the successful fusion independent of the restrictions of the sensors and provide accurate angles.\\
A Matlab model was produced which gets data from the Raspberry Pi via UDP network connection. Just a network cable needs to be connected between the host computer and the Raspberry Pi. On the Raspberry Pi is a C library used which enables the communication. With the help of this part the implementation can be tested and the configuration of the filter parameters can be improved.\\
Finally a doxygen file was written for automatic code documentation.

%- - - - - - - - - - - - - - - - - - - - - - 
% Section: Remaining project goals and outlook
%- - - - - - - - - - - - - - - - - - - - - - 
\section{Remaining project goals and outlook}
\label{sec:conclusion:outlook}

The basis of the PPM measurement is provided via a Kernel module. The analysis of the measured time differences needs to be done to get the separated control signals.\\
The orientation fusion delivers perfect representation of all three angles around the X,Y and Z axis of the system. Because of the usage of Euler angles there can be a gimbal lock when due to rotations two of the three axis fall together. Then one degree of freedom is lost. To solve this problem the system should not use Euler angles, instead Quaternion needs to be used.\\
Autonomous flying with just the orientation is not possible. Also the position of the Quadrocopter needs to be known. For this, additional sensor fusion focusing the velocity and position in all directions has to be calculated.\\
Last step is the implementation of the controller for the orientation and the position which finally controls the complete system. To switch between flying with the remote control and the autonomous flying a switch of the remote control needs to be used.