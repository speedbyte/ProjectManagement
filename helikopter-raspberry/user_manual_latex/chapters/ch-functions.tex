% Bsp. eines Hauptteils

\chapter{Functions in C}
\label{ch:functions}

These chapter is about the C-Code we developed to work via Raspberry Pi with the sensors.


\section{I$^2$C}

%-------------  write I2C 0 ---------------------%
\begin{lstlisting}[language=C, basicstyle=\small, caption=Write on I2C-1]
unsigned int g_lldI2c_WriteI2c_bl(unsigned char, const unsigned char*, unsigned int);
\end{lstlisting}


\begin{tabular}{lll}
Parameter: & unsigned char & slave address of device\\
					 & const unsigned char* & buffer with data to write\\
					 & unsigned int & number of bytes to write\\
					 &	\\
Return: & unsigned int & error detection, 0=OK 1=failure\\
							&&\\
Description: & \multicolumn{2}{l}{function to write several data on the I2C-1 Bus of the Raspberry PI}\\
\end{tabular}
\\

%-------------  read I2C 0 ---------------------%

\begin{lstlisting}[language=C, basicstyle=\small, caption=Read from I2C-1]
unsigned int g_lldI2c_ReadI2c_bl(unsigned char, const unsigned char*, unsigned int);
\end{lstlisting}

\begin{tabular}{lll}
Parameter: & unsigned char & slave address of device\\
					 & const unsigned char* & array to store read data\\
					 & unsigned int & number of bytes to read\\
					 &	\\
Return: & unsigned int & error detection, 0=OK 1=failure\\
							&&\\
Description: & \multicolumn{2}{l}{function to read several data on the I2C-1 Bus of the Raspberry PI}\\
\end{tabular}
\\

%-------------  write I2C 1 ---------------------%

\begin{lstlisting}[language=C, basicstyle=\small, caption=Write on I2C-0]
unsigned int g_lldI2c_WriteI2c0_bl(unsigned char, const unsigned char*, unsigned int);
\end{lstlisting}


\begin{tabular}{lll}
Parameter: & unsigned char & slave address of device\\
					 & const unsigned char* & buffer with data to write\\
					 & unsigned int & number of bytes to write\\
					 &	\\
Return: & unsigned int & error detection, 0=OK 1=failure\\
							&&\\
Description: & \multicolumn{2}{l}{function to write several data on the I2C-0 Bus of the Raspberry PI}\\
\end{tabular}
\\


%-------------  read I2C 1 ---------------------%

\begin{lstlisting}[language=C, basicstyle=\small, caption=Read from I2C-0]
unsigned int g_lldI2c_ReadI2c0_bl(unsigned char, const unsigned char*, unsigned int);
\end{lstlisting}

\begin{tabular}{lll}
Parameter: & unsigned char & slave address of device\\
					 & const unsigned char* & array to store read data\\
					 & unsigned int & number of bytes to read\\
					 &	\\
Return: & unsigned int & error detection, 0=OK 1=failure\\
							&&\\
Description: & \multicolumn{2}{l}{function to read several data on the I2C-0 Bus of the Raspberry PI}\\
\end{tabular}
\\





\subsection{Configuration}

$\#include <linux/i2c-dev.h>$ is doing the (local) device handling. We only use the slave address to communicate with the connected devices. Read or write commands are provided by fcntl.h.

\subsection{I$^2$C Write}

Writes the number of stated bytes from the write buffer with the "write" command to the chosen I$^2$c-device, depending on called function.

\subsection{I$^2$C Read}

Reads the number of stated bytes from the read buffer with the "read" command from the chosen I$^2$c-device, depending on called function.



\newpage
\section{Analog-Digital-Converter (ADC)}

\begin{lstlisting}[language=C, basicstyle=\small, caption=Read from ADC]
float g_halADC_get_ui16(unsigned char );
\end{lstlisting}


\begin{tabular}{lll}
Parameter: & unsigned char & A0-A3 input selection\\
					 &	\\
Return: & float & converted analog values\\
							&&\\
Description: & \multicolumn{2}{l}{Interface to read ADS1015 (ADC)}\\
\end{tabular}
\\


\subsection{Configuration}

Sensor Board Name: Pololu ADS1015\\
Sensor Name: GP2Y0A60SZLF\\

l\_mux\_ui8 = 0xC2;\\
" C "\textsubscript{16}:\\
The first Hex-Value depends on Starting Conversion + the Input, which Pin to read A0-3

" 2 "\textsubscript{16}: \\
The second Value is PGA (001)=+-4,099V and continuous Mode (0)\\


These three bytes are written to the ADS1015 to set the config register and start the conversion 

l\_writeBuf\_rg24[0] = 1;		\\ 
This sets the pointer register to write the following two bytes to the config register

l\_writeBuf\_rg24[1] = l\_mux\_ui8;   	\\ 
This sets the 8 MSBs of the config register (bits 15-8) to 11000011

l\_writeBuf\_rg24[2] = 0x23;  		\\ 
This sets the 8 LSBs of the config register (bits  7-0) to 00100011\\   


" 2 "\textsubscript{16}:\\
  // First Hex is sample Rate. (001) sets to 250SPS + Comp Mode (0)
	
" 3 "\textsubscript{16}:\\
  // Second Hex is Comp. config. (0011) disable the comparator
	
	
	
The l\_writeBuf\_rg24 is written to the configuration register of the ADC. After that, we can read the converted analog values from the chosen input.



\subsection{ADC Read}

To read a converted analog value, you need to set the pointer register to 0.\\

When the pointer register is set to 0 this signals that the converted analog value should be provided. You will get the these values when performing a i$^2$c-read command the next time. This value is 16 Bit large and will be calculated as a float-value, depending on the resolution which is adjusted.






\section{Infrared Sensor}

\begin{lstlisting}[language=C, basicstyle=\small, caption=Read Infrared]
float g_halADC_get_ui16(unsigned char );
\end{lstlisting}

\begin{tabular}{lll}
Parameter: & char & select Input on which IR is connected\\
					 &	\\
Return: & float & Voltage from Sensor\\
							&&\\
Description: & \multicolumn{2}{l}{Since our IR Sensor only provides analog output, we need to use the ADC}\\
\end{tabular}
\\



\subsection{Configuration}

There is no configuration of the Infrared sensor needed.

\subsection{Read Sensor Values}

We get the analog values with the ADC-Function.





\section{LIDAR-Lite Laser Sensor}








\begin{lstlisting}[language=C, basicstyle=\small, caption=get Laser Distance]
double g_LIDAR_getDistance_f64(void);
\end{lstlisting}

\begin{tabular}{lll}
Parameter: & void & \\
					 &	\\
Return: & double & distance in meter\\
							&&\\
Description: & \multicolumn{2}{l}{returns calculated distance value in meters}\\
\end{tabular}
\\

\begin{lstlisting}[language=C, basicstyle=\small, caption=trigger Laser measurement]
int g_LIDAR_readDistanceFromI2C_i32(void);
\end{lstlisting}

\begin{tabular}{lll}
Parameter: & void & \\
					 &	\\
Return: & int & error detection, 0=OK -1=failure\\
							&&\\
Description: & \multicolumn{2}{l}{triggers a measurement and stores the result}\\
\end{tabular}
\\


\subsection{Configuration}

Trigger Measurement of Distance (DC stabilization cycle, Signal Acquisition, DataProcessing)\\

First Config Byte: 0x00;\\
Representation of configuration register 0x00 of the laser sensor.\\
Second Config Byte: 0x04;\\
Take acquisition and correlation processing with DC correction\\

Set Reg 0x8f as Output-Register to read a two-byte value which gives the distance in cm.\\

\subsection{Read Sensor Values}

Read two-byte distance in cm from register 0x8f. This value is stored as a decimal value in cm.\\
Alternative you can read the high-byte of the measured value from register 0x0f and the low-byte from register 0x10.\\


%\chapter{Ergebnisse}
%\label{sec:ergeb}
%\enquote{Neuigkeiten} Messergebnisse
%
%\section{Literaturverweise}
%\label{sec:real-literatur}
%
%Verweise im Text: \cite{doc:stz} und \cite{doc:gun}.


